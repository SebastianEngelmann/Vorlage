\documentclass[%
	paper=A4,				% paper size --> A4 is default in Germany
	%twoside=true,				% onesite or twoside printing
	openright,				% doublepage cleaning ends up right side
	parskip=full,				% spacing value / method for paragraphs
	chapterprefix=true,			% prefix for chapter marks
	11pt,					% font size
	headings=normal,			% size of headings
	bibliography=totoc,			% include bib in toc
	%listof=totoc, 				% include listof entries in toc
	titlepage=on,				% own page for each title page
	captions=tableabove,			% display table captions above the float env
	draft=false,				% value for draft version
]{scrartcl}%

% Packages
% -----------------------------------
\usepackage[utf8]{inputenc}		
\usepackage[T1]{fontenc}
\usepackage[ngerman]{babel}
\usepackage[style=authoryear, backend=biber]{biblatex}
\usepackage{csquotes}
\bibliography{./Literatur/quellen.bib}					
\usepackage{mdwlist}
\usepackage{placeins}
\usepackage{graphicx} 				% Grafiken einfügen (pdf,png - aber jpg vermeiden)
\graphicspath{{./Bilder/}}              	% Pfad zu den Bildern
\usepackage{url}				% URL's formatieren (z.B. in Literatur) 
\usepackage{tabularx} 				% bessere Gestaltung von Tabellen
\usepackage{longtable, lscape} 				 
\usepackage[active]{srcltx}
\usepackage{listings}
\usepackage{setspace} 				% Zeileneinstellung
\newtheorem{mydef}{Merksatz}  			% Falls Beispiele, Merksätze m. fortl. Nr. gebr. werden
\usepackage{amsmath}
\usepackage{amsthm}
\usepackage{amssymb}
%\usepackage{amsfonts}
\usepackage{pdfpages}
\newtheorem{bsp}{Beispiel}
%\usepackage{indentfirst}
\rmfamily 					% Serifenschrift
%\usepackage{unicode-math}
%\usepackage{libertine}
%\setmathfont{Asana Math}
\usepackage{mathrsfs}
\usepackage[nooneline]{caption}
\usepackage{float}
\usepackage{fancyvrb}
\usepackage{ragged2e}
%\usepackage{multirow}
\numberwithin{equation}{section}
% Literatur in Literaturverzeichnis umbenennen
\addto\captionsngerman{\renewcommand{\refname}{Literaturverzeichnis}}

\definecolor{mygray}{gray}{.4}
\definecolor{light-gray}{gray}{0.95}

\lstdefinelanguage{Renhanced}[]{R},%
  alsoother={:_\$}}
  
\lstset{language = Renhanced,
	basicstyle = \ttfamily\color{black}, %usepackage courier muss geladen werden
	stringstyle = \color{mygray},
    commentstyle = \color{mygray}\textit,
    keywordstyle = \bfseries,
	numbers=left,
	numbersep=5pt,
	numberstyle = \footnotesize,
    breaklines=true,
    backgroundcolor=\color{light-gray},
    keepspaces = true,
    flexiblecolumns = true,
    showtabs = false,    
    showstringspaces=false
    }


%%%%%%%%%%%%%%%%%%%%%%%%%%%%%%%%%%%%%%%%%%%%%%%%%%%%%%%%%%%%%%%%%%%%
%%%%%%%%%%%%%%%%%%%%%  Definitionen  %%%%%%%%%%%%%%%%%%%%%%%%%%%%%%% 
%%%%%%%%%%%%%%%%%%%%%%%%%%%%%%%%%%%%%%%%%%%%%%%%%%%%%%%%%%%%%%%%%%%%
\newtheorem{theorem}{Theorem}[section]
\newtheorem{lemma}[theorem]{Lemma}
\newtheorem{bemerkung}[theorem]{Bemerkung}
\newtheorem*{definition}{Definition}
\newtheorem{notation}[theorem]{Notation}
\newtheorem{korollar}[theorem]{Korollar}
\newtheorem{proposition}[theorem]{Proposition}

\def\Q{\mathbb{Q}}
\def\F{\mathbb{F}}
\def\Z{\mathbb{Z}}
\def\N{\mathbb{N}}
\def\R{\mathbb{R}}

\setlength{\parindent}{1.5em} % z. B. 0pt für den Einzug
\setlength{\parskip}{1.5em}  % z. B. 0pt


\usepackage{hyperref} 				% für Hyperlinks in PDF-Dokumenten     

% Document
% -----------------------------------
\begin{document}
%\frontmatter 
\pagenumbering{Roman}
    %!TEX root = ../main.tex

% Titelseite soll keine Kopf oder Fußzeile haben
\thispagestyle{empty}

% Alle Elemente sollen zentriert sein
\begin{center}

\vspace*{-20mm}

{\LARGE Lehrstuhl für tba\\[1mm]}
der Freien Universität Berlin\\

\vspace*{1cm}

\includegraphics[width=0.18\textwidth]{fu_logo}

\vspace*{1cm}

% Art der Arbeit => (Bachelorarbeit ,Diplomarbeit, Masterarbeit, Seminararbeit)
{\Large \textbf{Bachelorarbeit}}\\ 

\vspace{1cm}

% Titel der Arbeit 
{\Large \textbf{Hier fogt der Titel}}\\ 
\vspace*{1mm}
{\Large \textbf{dieser kann auf bis zu drei}}\\ 
\vspace*{1mm}
{\Large \textbf{Zeilen verteilt werden wenn nötig}}\\
\vspace*{2mm}
%--\\
%\vspace*{2mm}
%{\Large \textbf{Hier gehts vielleicht noch weiter}}\\

\vspace{1.5cm}

% Name des/der Autors/Autoren
{\LARGE Vorname Nachname}\\[25mm]

% Gutachter, Kontaktdaten und Abgabetermin
\parbox{120mm}{
\begin{large}
\begin{tabbing}
Gutachter(in): \hspace{.7cm} \= Prof. Dr. Vorname Nachname \\[4mm]
Verfasser:\> Vorname Nachname\\ % alphabetische Reihenfolge (Nachname)
Matrikel-Nr.:\> tba\\
Adresse:\> tba\\
Email:\> tba\\
Telefon:\> tba\\[8mm]
\textbf{Abgabetermin:} \> \textbf{08. August 2014}\\
\end{tabbing}
\end{large}
}

\end{center}
 			% Titelblatt
    %!TEX root = ../main.tex

\newpage

\vspace*{1cm}

\begin{center}
    \textbf{Abstract}
\end{center}

\vspace*{1cm}

\noindent % 12 Zeilen 

tba   



 		% Abstract auf Deutsch 
    %%!TEX root = ../main.tex

\newpage

\vspace*{1cm}

\begin{center}
    \textbf{Abstract}
\end{center}

\vspace*{1cm}

\noindent 

tba
 		% Abstract in English
    \onehalfspacing                  		% Zeilenabstand ab hier 1.5 zeilig
    \tableofcontents 				% Inhaltsverzeichnis
    \newpage
    \listoffigures 			 	% Abbildungsverzeichnis
    \newpage
    \listoftables				% Tabellenverzeichnis
    \newpage
% -----------------------------------
%\mainmatter 		
%\pagenumbering{arabic}				% die einzelnen Kapitel
    %!TEX root = ../main.tex

\pagenumbering{arabic}

\FloatBarrier
\section{tba}

tba

    \input{./Kapitel/kapitel2}
    \input{./Kapitel/kapitel3}
    \input{./Kapitel/kapitel4}
    %!TEX root = ../main.tex

\FloatBarrier
\section{tba}

tba


% -----------------------------------
%\backmatter 
\begin{appendix}
%\input{./Anhang/} 				% Pfad & Datei einfuegen => ./Anhang/*.tex 
\end{appendix}
\RaggedRight
\clearpage
\begingroup
%\RaggedRight 
\printbibliography				% Literaturquellen einbinden
\endgroup
%!TEX root = /Users/sebastianengelmann/Documents/Studium/Bachelorarbeit/Bachelor_Arbeit/4547702.tex

\newpage
\thispagestyle{empty}

\begin{large}

\vspace*{2cm}

\noindent
Hiermit versichere ich, dass ich die vorliegende Arbeit
selbst\"andig verfasst und keine anderen als die angegebenen Quellen
und Hilfsmittel verwendet habe.

\vspace{2cm}

\noindent
Berlin, den 08. August 2014

\vspace{3cm}

\hspace*{7cm}%
\dotfill\\
\hspace*{8.5cm}%
\textit{(Unterschrift des Verfassers)}

\end{large}
 			% Eidesstattliche Erklärung (nicht bei Seminararb.)

\end{document}
